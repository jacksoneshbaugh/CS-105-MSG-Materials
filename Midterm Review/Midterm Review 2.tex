\documentclass[11pt]{exam}
\newcommand{\myname}{Jackson Eshbaugh}
\newcommand{\myemail}{eshbaugj}
\newcommand{\myhwtype}{Midterm Review}
\newcommand{\myhwnum}{2}
\newcommand{\myclass}{CS 105}
\newcommand{\mylecture}{1}
\newcommand{\mysection}{}

% Prefix for numedquestion's
\newcommand{\questiontype}{Question}

% Use this if your "written" questions are all under one section
% For example, if the homework handout has Section 5: Written Questions
% and all questions are 5.1, 5.2, 5.3, etc. set this to 5
% Use for 0 no prefix. Redefine as needed per-question.
\newcommand{\writtensection}{0}

\usepackage{amsmath, amsfonts, amsthm, amssymb}  % Some math symbols
\usepackage{enumerate}
\usepackage{enumitem}
\usepackage{graphicx}
\usepackage{hyperref}
\usepackage[all]{xy}
\usepackage{wrapfig}
\usepackage{fancyvrb}
\usepackage[T1]{fontenc}
\usepackage{float}
\usepackage{listings}
\usepackage{booktabs}
\usepackage{framed}
\usepackage{parcolumns}

\usepackage{centernot}
\usepackage{mathtools}
\DeclarePairedDelimiter{\ceil}{\lceil}{\rceil}
\DeclarePairedDelimiter{\floor}{\lfloor}{\rfloor}
\DeclarePairedDelimiter{\card}{\vert}{\vert}

% Uncomment the following line to get Solarized-themed source listings
% You will have had to already installed the solarized-light package
% https://github.com/jez/latex-solarized
%
%\usepackage{solarized-light}

\setlength{\parindent}{0pt}
\setlength{\parskip}{5pt plus 1pt}
\pagestyle{empty}

\def\indented#1{\list{}{}\item[]}
\let\indented=\endlist

\newcounter{questionCounter}
\newcounter{partCounter}[questionCounter]

\newenvironment{namedquestion}[1][\arabic{questionCounter}]{%
    \addtocounter{questionCounter}{1}%
    \setcounter{partCounter}{0}%
    \vspace{.2in}%
    \noindent{\bf #1}%
    \vspace{0.3em} \hrule \vspace{.1in}%
}{}

\newenvironment{numedquestion}[0]{%
    \stepcounter{questionCounter}%
    \vspace{.2in}%
    \ifx
        \writtensection\undefined
        \noindent{\bf \questiontype \; \arabic{questionCounter}. }%
    \else
        \if
            \writtensection0
            \noindent{\bf \questiontype \; \arabic{questionCounter}. }%
        \else
            \noindent{\bf \questiontype \; \writtensection.\arabic{questionCounter} }%
        \fi
        \vspace{0.3em} \hrule \vspace{.1in}%
        }{}

\newenvironment{alphaparts}[0]{%
    \begin{enumerate}[label=\textbf{(\alph*)}]
    }{\end{enumerate}}

\newenvironment{arabicparts}[0]{%
    \begin{enumerate}[label=\textbf{\arabic{questionCounter}.\arabic*})]
    }{\end{enumerate}}

\newenvironment{questionpart}[0]{%
    \item
    }{}

\newcommand{\answerbox}[1]{
    \begin{framed}
    \vspace{#1}
    \end{framed}}

\pagestyle{head}

\headrule
\header{\textbf{\myclass\ \mylecture\mysection}}%
{\textbf{\myname\ }}%
{\textbf{\myhwtype\ \myhwnum}}

\begin{document}
    \thispagestyle{plain}
    \begin{center}
    {\Large \myclass{} \myhwtype{} \myhwnum}
        \\
        \myname{}\\
        \today
    \end{center}

    \begin{numedquestion}
        For each of the following, take the English and convert it into Processing conditionals.
        Anything written in \texttt{teletype text} is a variable or function that you can use.
        \begin{alphaparts}

            \item If the mouse is in the box (given by boolean \texttt{mouseInBox}), then make the box rainbow-colored (using \texttt{rainbow()}, a function that was implemented for you for this task).
            \answerbox{3cm}
            \item If the \texttt{float temperature} is below 32, print "Freezing."
            If it's between 32 and 50, print "Cold." Otherwise, print "Warm."
            \answerbox{3cm}
            \item If the mouse is in the box (\texttt{x} = 50, \texttt{y} = 25, \texttt{width} = 75, \texttt{height} = 80), draw a point at the mouse's location (assume all Processing variables are available).
            \answerbox{3cm}
        \end{alphaparts}
    \end{numedquestion}
    \pagebreak
    \begin{numedquestion}
        Consider the following code.
        Respond to each part below.
        \begin{lstlisting}[language=Java]
            int score = 85;
            if (score >= 90) {
              println("A");
            } else if (score >= 80) {
              println("B");
            } else {
              println("C");
            }
        \end{lstlisting}

        \begin{alphaparts}
            \item What is the output of the code?
            \vspace{3cm}
            \item If the first line was changed to \texttt{int score = 46;}, what would the output be?
            \vspace{3cm}
            \item If the first line was changed to \texttt{int score = 97;}, what would the output be?
            \vspace{3cm}
            \item If the first line was changed to \texttt{int score = true;}, what would the output be?
            \vspace{3cm}
        \end{alphaparts}
    \end{numedquestion}
    \pagebreak
    \begin{numedquestion}
        Consider the following code.
        Respond to each part below.
        \begin{lstlisting}[language=Java]
            int a = 35;

            if(a % 2 == 0) {
            println("Even!");
            } else {
                println("Odd!");
                if(a % 5 == 0) {
                    println("5!");
                }
            }
        \end{lstlisting}

        \begin{alphaparts}
            \item What is the output of the code?
            \vspace{2cm}
            \item If the first line was changed to \texttt{int a = 23;}, what would the output be?
            \vspace{2cm}
            \item If the first line was changed to \texttt{int a = 72;}, what would the output be?
            \vspace{2cm}
            \item If the first line was changed to \texttt{int a = 50;}, what would the output be?
            \vspace{2cm}
        \end{alphaparts}
    \end{numedquestion}

    \pagebreak
    \begin{numedquestion}
        For each of the following, determine if the loop runs 0 times, a finite number of times, or an infinite number of times.
        \begin{parcolumns}[colwidths={1=.6\textwidth, 2=.35\textwidth}]{2}
            \colchunk{
                \begin{lstlisting}[language=Java]
                    int i = 0;
                    while(i < 5) {
                        println(i);
                        i--;
                        i++;
                    }
                \end{lstlisting}
            }
            \colchunk{
                \begin{choices}
                    \choice 0 times
                    \choice finite number of times
                    \choice infinite number of times
                \end{choices}
            }
        \end{parcolumns}
        \vspace{2cm}
        \begin{parcolumns}[colwidths={1=.6\textwidth, 2=.35\textwidth}]{2}
            \colchunk{
                \begin{lstlisting}[language=Java]
                    while(false){
                        println("yay!");
                      }
                \end{lstlisting}
            }
            \colchunk{
                \begin{choices}
                    \choice 0 times
                    \choice finite number of times
                    \choice infinite number of times
                \end{choices}
            }
        \end{parcolumns}
        \vspace{2cm}
        \begin{parcolumns}[colwidths={1=.6\textwidth, 2=.35\textwidth}]{2}
            \colchunk{
                \begin{lstlisting}[language=Java]
                    int i = 0;
                      while(i < 5) {
                        println(i);
                        i += 2;
                      }
                \end{lstlisting}
            }
            \colchunk{
                \begin{choices}
                    \choice 0 times
                    \choice finite number of times
                    \choice infinite number of times
                \end{choices}
            }
        \end{parcolumns}
        \vspace{2cm}
        \begin{parcolumns}[colwidths={1=.6\textwidth, 2=.35\textwidth}]{2}
            \colchunk{
                \begin{lstlisting}[language=Java]
                    int i = 0;
                      while(i > -10) {
                        println(i);
                        i -= 2;
                      }
                \end{lstlisting}
            }
            \colchunk{
                \begin{choices}
                    \choice 0 times
                    \choice finite number of times
                    \choice infinite number of times
                \end{choices}
            }
        \end{parcolumns}
    \end{numedquestion}

    \begin{numedquestion}
        For each of the following loops, write a for loop that performs the exact same tasks as the given loop.
        \begin{lstlisting}[language=Java]
                      int i = 0;
                      while(i < 5) {
                        println(i);
                        rect(width/2 * i, height/2 * i, 15, 15);
                        i += 2;
                      }
        \end{lstlisting}
        \vspace{.5cm}
        \answerbox{3cm}
        \vspace{.5cm}
        \begin{lstlisting}[language=Java]
                      int i = 0;
                      while(i > - 10) {
                        println(i);
                        mySuperAwesomeCoolFunction3(i);
                        i -= 2;
                      }
        \end{lstlisting}
        \vspace{.5cm}
        \answerbox{3cm}
    \end{numedquestion}
    \pagebreak
    \begin{numedquestion}
        Let's trace some loops!
        For each loop given below, give its output, along with the final values of all variables.
        \begin{lstlisting}[language=Java]
            int total = 0;
            for (int i = 1; i <= 5; i++) {
              if (i % 2 == 0) {
                total += i;
              }
              println("i = " + i + ", total = " + total);
            }
        \end{lstlisting}
        \vspace{.5cm}
        \begin{parcolumns}[1=.75\textwidth, 2=.25\textwidth]{2}
            \colchunk{
                \textbf{Console}
                \answerbox{4cm}
            }
            \colchunk{
                \textbf{Variables}
                \answerbox{4cm}
            }
        \end{parcolumns}
        \vspace{1cm}
        \begin{lstlisting}[language=Java]
            for (int i = 1; i <= 5; i++) {
              for (int j = 1; j <= i; j++) {
                print(j + " ");
              }
              println();  // new line
            }
        \end{lstlisting}
        \vspace{1cm}
        \begin{parcolumns}[1=.75\textwidth, 2=.25\textwidth]{2}
            \colchunk{
                \textbf{Console}
                \answerbox{4cm}
            }
            \colchunk{
                \textbf{Variables}
                \answerbox{4cm}
            }
        \end{parcolumns}
    \end{numedquestion}

\end{document}