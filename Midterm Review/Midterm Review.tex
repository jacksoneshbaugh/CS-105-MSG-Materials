\documentclass[11pt]{exam}
\newcommand{\myname}{Jackson Eshbaugh}
\newcommand{\myemail}{jezimmer}
\newcommand{\myhwtype}{Midterm Review}
\newcommand{\myhwnum}{1}
\newcommand{\myclass}{CS 105}
\newcommand{\mylecture}{1}
\newcommand{\mysection}{}

% Prefix for numedquestion's
\newcommand{\questiontype}{Question}

% Use this if your "written" questions are all under one section
% For example, if the homework handout has Section 5: Written Questions
% and all questions are 5.1, 5.2, 5.3, etc. set this to 5
% Use for 0 no prefix. Redefine as needed per-question.
\newcommand{\writtensection}{0}

\usepackage{amsmath, amsfonts, amsthm, amssymb}  % Some math symbols
\usepackage{enumerate}
\usepackage{enumitem}
\usepackage{graphicx}
\usepackage{hyperref}
\usepackage[all]{xy}
\usepackage{wrapfig}
\usepackage{fancyvrb}
\usepackage[T1]{fontenc}
\usepackage{float}
\usepackage{listings}
\usepackage{booktabs}

\usepackage{centernot}
\usepackage{mathtools}
\DeclarePairedDelimiter{\ceil}{\lceil}{\rceil}
\DeclarePairedDelimiter{\floor}{\lfloor}{\rfloor}
\DeclarePairedDelimiter{\card}{\vert}{\vert}

% Uncomment the following line to get Solarized-themed source listings
% You will have had to already installed the solarized-light package
% https://github.com/jez/latex-solarized
%
%\usepackage{solarized-light}

\setlength{\parindent}{0pt}
\setlength{\parskip}{5pt plus 1pt}
\pagestyle{empty}

\def\indented#1{\list{}{}\item[]}
\let\indented=\endlist

\newcounter{questionCounter}
\newcounter{partCounter}[questionCounter]

\newenvironment{namedquestion}[1][\arabic{questionCounter}]{%
    \addtocounter{questionCounter}{1}%
    \setcounter{partCounter}{0}%
    \vspace{.2in}%
    \noindent{\bf #1}%
    \vspace{0.3em} \hrule \vspace{.1in}%
}{}

\newenvironment{numedquestion}[0]{%
    \stepcounter{questionCounter}%
    \vspace{.2in}%
    \ifx\writtensection\undefined
    \noindent{\bf \questiontype \; \arabic{questionCounter}. }%
    \else
        \if\writtensection0
        \noindent{\bf \questiontype \; \arabic{questionCounter}. }%
        \else
            \noindent{\bf \questiontype \; \writtensection.\arabic{questionCounter} }%
        \fi
        \vspace{0.3em} \hrule \vspace{.1in}%
        }{}

\newenvironment{alphaparts}[0]{%
    \begin{enumerate}[label=\textbf{(\alph*)}]
    }{\end{enumerate}}

\newenvironment{arabicparts}[0]{%
    \begin{enumerate}[label=\textbf{\arabic{questionCounter}.\arabic*})]
    }{\end{enumerate}}

\newenvironment{questionpart}[0]{%
    \item
    }{}

\newcommand{\answerbox}[1]{
    \begin{framed}
    \vspace{#1}
    \end{framed}}

\pagestyle{head}

\headrule
\header{\textbf{\myclass\ \mylecture\mysection}}%
{\textbf{\myname\ }}%
{\textbf{\myhwtype\ \myhwnum}}

\begin{document}
    \thispagestyle{plain}
    \begin{center}
    {\Large \myclass{} \myhwtype{} \myhwnum} \\
    \myname{}\\
    \today
    \end{center}

    \begin{numedquestion}
        What types of variables are there?
        \\
        \\
        \\
        \\
        \\
    \end{numedquestion}

    \begin{numedquestion}
        Let's use \texttt{==} and \texttt{.equals()}.

        \begin{alphaparts}
            \item Are integers \texttt{x} and \texttt{y} equal?
            \\
            \\
            \item Are Strings \texttt{str1} and \texttt{str2} equal?
            \\
            \\
        \end{alphaparts}
    \end{numedquestion}

    \begin{numedquestion}
        \begin{alphaparts}
            \item Convert the statement "it is not raining outside" to a \texttt{boolean} in Java.
            \\
            \\
            \item Convert the statement "it is raining outside" to a \texttt{boolean} in Java.
            \\
            \\
        \end{alphaparts}
    \end{numedquestion}

    \pagebreak
    \begin{numedquestion}
        Complete the truth table for \texttt{a \&\& b}.
        \begin{table}[H]
            \centering
            \renewcommand{\arraystretch}{1.5}
            \begin{tabular}{l l l}
                \toprule
                \texttt{a}     & \texttt{b}     & \texttt{a \&\& b} \\ \midrule
                \texttt{true}  & \texttt{true}  &                   \\
                \texttt{true}  & \texttt{false} &                   \\
                \texttt{false} & \texttt{true}  &                   \\
                \texttt{false} & \texttt{false} &                   \\
                \bottomrule
            \end{tabular}
        \end{table}
    \end{numedquestion}

    \begin{numedquestion}
        Complete the truth table for \texttt{a || b}.
        \begin{table}[H]
            \centering
            \renewcommand{\arraystretch}{1.5}
            \begin{tabular}{l l l}
                \toprule
                \texttt{a}              & \texttt{b}              & \texttt{a || b} \\ \midrule
                \texttt{true}  & \texttt{true}  &                   \\
                \texttt{true}  & \texttt{false} &                   \\
                \texttt{false} & \texttt{true}  &                   \\
                \texttt{false} & \texttt{false} &                   \\
                \bottomrule
            \end{tabular}
        \end{table}
    \end{numedquestion}

    \begin{numedquestion}
        Complete the truth table for \texttt{!a}.
        \begin{table}[H]
            \centering
            \renewcommand{\arraystretch}{1.5}
            \begin{tabular}{l l}
                \toprule
                \texttt{a}     & \texttt{!a} \\ \midrule
                \texttt{true}  &              \\
                \texttt{false} &              \\
                \bottomrule
            \end{tabular}
        \end{table}
    \end{numedquestion}

    \begin{numedquestion}
        For each snippet of code, what is the output?
        \begin{lstlisting}[language=Java]
            boolean a = true;
            boolean b = false;
            println(!(a || b) && (a && !b));
        \end{lstlisting}
        \begin{lstlisting}[language=Java]
            boolean x = false;
            boolean y = true;
            boolean z = false;
            println((x || y) && !(y && z) || (!x && z));
        \end{lstlisting}
    \end{numedquestion}

    \begin{numedquestion}
        For each of the following, write the function header.
        \begin{alphaparts}
            \item Write a function that takes two integers and returns \texttt{true} if both integers are either positive or both are negative, and \texttt{false} otherwise.
            \\
            \\
            \item Write a function \texttt{isLeapYear} that takes an integer year and returns \texttt{true} if it is a leap year, and \texttt{false} otherwise.
            \\
            \\
            \item Write a function \textt{inRange} that takes two integers. Return true if one of the two numbers is in the range 10-20 (inclusive), but not both.
            \\
            \\
        \end{alphaparts}
    \end{numedquestion}

    \begin{numedquestion}
        For 8 (a) and (b), create a ground truth table that has a good range of test cases.
        \begin{alphaparts}
            \item Question 8 (a)
            \\
            \\
            \\
            \\
            \\
            \\
            \item Question 8 (c)
        \end{alphaparts}
    \end{numedquestion}
    \pagebreak
    \begin{numedquestion}
        Implement the function from either question 8 (a) or (c).
        Do not do this on the computer, instead use paper and pencil (to better prepare for the exam).
        \\
        \\
        \\
        \\
        \\
        \\
        \\
        \\
        \\
        \\
        \\
        \\
        \\
        \\
        \\
        \\
        \\
        \\
        \\
        \\
        \\
        \\
        \\
        \\
        \\
        \\
        \\
        \\
        \\
        \\
        \\
        \\
        \\
        \\
        \\
        \\
        \\
    \end{numedquestion}

    Now that we've reviewed together, use the old midterms (on Moodle) to practice.
    Ideally, we'll review those that you have already completed but have questions about.
    Also, if you have anything you'd like me to hit on Sunday, \textbf{\textit{please let me know!}}
    I already plan to cover conditionals, loops, and maybe some more functions, but please either tell me here or email me (\textit{eshbaugj@lafayette.edu}) if you want me to touch on anything specifically on Sunday.

\end{document}