\documentclass[11pt]{exam}
\newcommand{\myname}{Jackson Eshbaugh}
\newcommand{\myemail}{eshbaugj}
\newcommand{\myhwtype}{MSG Activity}
\newcommand{\myhwnum}{1}
\newcommand{\myclass}{CS 105}
\newcommand{\mylecture}{1}
\newcommand{\mysection}{}

% Prefix for numedquestion's
\newcommand{\questiontype}{Question}

% Use this if your "written" questions are all under one section
% For example, if the homework handout has Section 5: Written Questions
% and all questions are 5.1, 5.2, 5.3, etc. set this to 5
% Use for 0 no prefix. Redefine as needed per-question.
\newcommand{\writtensection}{0}

\usepackage{amsmath, amsfonts, amsthm, amssymb}  % Some math symbols
\usepackage{enumerate}
\usepackage{enumitem}
\usepackage{graphicx}
\usepackage{hyperref}
\usepackage[all]{xy}
\usepackage{wrapfig}
\usepackage{fancyvrb}
\usepackage[T1]{fontenc}
\usepackage{float}
\usepackage{listings}
\usepackage{booktabs}
\usepackage{framed}
\usepackage{parcolumns}

\usepackage{centernot}
\usepackage{mathtools}
\DeclarePairedDelimiter{\ceil}{\lceil}{\rceil}
\DeclarePairedDelimiter{\floor}{\lfloor}{\rfloor}
\DeclarePairedDelimiter{\card}{\vert}{\vert}

% Uncomment the following line to get Solarized-themed source listings
% You will have had to already installed the solarized-light package
% https://github.com/jez/latex-solarized
%
%\usepackage{solarized-light}

\setlength{\parindent}{0pt}
\setlength{\parskip}{5pt plus 1pt}
\pagestyle{empty}

\def\indented#1{\list{}{}\item[]}
\let\indented=\endlist

\newcounter{questionCounter}
\newcounter{partCounter}[questionCounter]

\newenvironment{namedquestion}[1][\arabic{questionCounter}]{%
    \addtocounter{questionCounter}{1}%
    \setcounter{partCounter}{0}%
    \vspace{.2in}%
    \noindent{\bf #1}%
    \vspace{0.3em} \hrule \vspace{.1in}%
}{}

\newenvironment{numedquestion}[0]{%
    \stepcounter{questionCounter}%
    \vspace{.2in}%
    \ifx
        \writtensection\undefined
        \noindent{\bf \questiontype \; \arabic{questionCounter}. }%
    \else
        \if
            \writtensection0
            \noindent{\bf \questiontype \; \arabic{questionCounter}. }%
        \else
            \noindent{\bf \questiontype \; \writtensection.\arabic{questionCounter} }%
        \fi
        \vspace{0.3em} \hrule \vspace{.1in}%
        }{}

\newenvironment{alphaparts}[0]{%
    \begin{enumerate}[label=\textbf{(\alph*)}]
    }{\end{enumerate}}

\newenvironment{arabicparts}[0]{%
    \begin{enumerate}[label=\textbf{\arabic{questionCounter}.\arabic*})]
    }{\end{enumerate}}

\newenvironment{questionpart}[0]{%
    \item
    }{}

\newcommand{\answerbox}[1]{
    \begin{framed}
    \vspace{#1}
    \end{framed}}

\pagestyle{head}

\headrule
\header{\textbf{\myclass\ \mylecture\mysection}}%
{\textbf{\myname\ }}%
{\textbf{\myhwtype\ \myhwnum}}

\begin{document}
    \thispagestyle{plain}
    \begin{center}
    {\Large \myclass{} \myhwtype{} \myhwnum}:
        \\
        {\Large Arrays}
        \\
        \myname{}\\
        \today
    \end{center}

    \begin{numedquestion}
    	What is an array? What types can it hold? Is it variable- or fixed-length?
    	
    	\answerbox{3cm}
    \end{numedquestion}
    
	% model an array    
    
    \begin{numedquestion}
    	If the pixel color values of the screen in Processing were stored in a (2D) array, what would the dimensions of the array be?
    	\begin{alphaparts}
    		\item \texttt{width} = 
    		\item \texttt{height} =
    	\end{alphaparts}
    	
    	What type would the array be?
    	\vspace{1cm}
    	
    	How would I set the pixel at $(2, 4)$ to be green?
    	\answerbox{1cm}
    	
    	How would I set the pixel at $(i, j)$ to be a \texttt{color} \texttt{col}?\\
    	\textit{(this is the general solution to this kind of problem)}
    	\answerbox{1cm}
    \end{numedquestion}
    
    % matrix (lin alg)
    % print contents of array
    
    \begin{numedquestion}
    	How would we print the contents of an array? For this example, assume we're printing an array \texttt{arr} of \texttt{integer} values. Write a function \texttt{printIntArray} that accomplishes this task.
    	\answerbox{4cm}
    \end{numedquestion}
    
    \begin{numedquestion}
    	Write a function that calculates the mean of an array of floats and returns it.
    	\answerbox{4cm}
    \end{numedquestion}

\end{document}