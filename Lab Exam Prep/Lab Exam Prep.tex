\documentclass[11pt]{exam}
\newcommand{\myname}{Jackson Eshbaugh}
\newcommand{\myemail}{eshbaugj@lafayette.edu}
\newcommand{\myhwtype}{Lab Exam Practice}
\newcommand{\myhwnum}{1}
\newcommand{\myclass}{CS 105}
\newcommand{\mylecture}{1}
\newcommand{\mysection}{}

% Prefix for numedquestion's
\newcommand{\questiontype}{Question}

% Use this if your "written" questions are all under one section
% For example, if the homework handout has Section 5: Written Questions
% and all questions are 5.1, 5.2, 5.3, etc. set this to 5
% Use for 0 no prefix. Redefine as needed per-question.
\newcommand{\writtensection}{0}

\usepackage{amsmath, amsfonts, amsthm, amssymb}  % Some math symbols
\usepackage{enumerate}
\usepackage{enumitem}
\usepackage{graphicx}
\usepackage{hyperref}
\usepackage[all]{xy}
\usepackage{wrapfig}
\usepackage{fancyvrb}
\usepackage[T1]{fontenc}
\usepackage{float}
\usepackage{listings}
\usepackage{booktabs}

\usepackage{centernot}
\usepackage{mathtools}
\DeclarePairedDelimiter{\ceil}{\lceil}{\rceil}
\DeclarePairedDelimiter{\floor}{\lfloor}{\rfloor}
\DeclarePairedDelimiter{\card}{\vert}{\vert}

% Uncomment the following line to get Solarized-themed source listings
% You will have had to already installed the solarized-light package
% https://github.com/jez/latex-solarized
%
%\usepackage{solarized-light}

\setlength{\parindent}{0pt}
\setlength{\parskip}{5pt plus 1pt}
\pagestyle{empty}

\def\indented#1{\list{}{}\item[]}
\let\indented=\endlist

\newcounter{questionCounter}
\newcounter{partCounter}[questionCounter]

\newenvironment{namedquestion}[1][\arabic{questionCounter}]{%
    \addtocounter{questionCounter}{1}%
    \setcounter{partCounter}{0}%
    \vspace{.2in}%
    \noindent{\bf #1}%
    \vspace{0.3em} \hrule \vspace{.1in}%
}{}

\newenvironment{numedquestion}[0]{%
    \stepcounter{questionCounter}%
    \vspace{.2in}%
    \ifx\writtensection\undefined
    \noindent{\bf \questiontype \; \arabic{questionCounter}. }%
    \else
        \if\writtensection0
        \noindent{\bf \questiontype \; \arabic{questionCounter}. }%
        \else
            \noindent{\bf \questiontype \; \writtensection.\arabic{questionCounter} }%
        \fi
        \vspace{0.3em} \hrule \vspace{.1in}%
        }{}

\newenvironment{alphaparts}[0]{%
    \begin{enumerate}[label=\textbf{(\alph*)}]
    }{\end{enumerate}}

\newenvironment{arabicparts}[0]{%
    \begin{enumerate}[label=\textbf{\arabic{questionCounter}.\arabic*})]
    }{\end{enumerate}}

\newenvironment{questionpart}[0]{%
    \item
    }{}

\newcommand{\answerbox}[1]{
    \begin{framed}
    \vspace{#1}
    \end{framed}}

\pagestyle{head}

\headrule
\header{\textbf{\myclass\ \mylecture\mysection}}%
{\textbf{\myname\ }}%
{\textbf{\myhwtype\ \myhwnum}}

\begin{document}
    \thispagestyle{plain}
    \begin{center}
    {\Large \myclass{} \myhwtype{} \myhwnum} \\
    \myname{}\\
    \today
    \end{center}

    \begin{numedquestion}
        \textbf{Roll Call:}
        The CS department is hosting a picnic (with extra credit for whoever attends).
        There's one problem: they don't have a way to track attendance for the extra credit.
        Professor Frank (department) recruited you to build a Processing program to do this task.

        \enumerate
            \item Create a \texttt{Student} class that holds a student's name, age, L-number, major, and minor.
                  Implement a constructor that sets each of these values, and a method that gets each of them.

            \item Create an array of \texttt{Student} objects to track attendance.
                  Make it 100 elements long—we're not that big of a department).

            \item Implement the following:
                \begin{alphaparts}
                    \item \texttt{add()}: add a student to the array
                    \item \texttt{remove()}: take a student out of the
                    \item \texttt{find()}: check if this specific student is in the array (should get extra credit)
                    \item \texttt{display()}: display the array in the console (in some logical way).
                \end{alphaparts}
    \end{numedquestion}
    \pagebreak
    \begin{numedquestion}
        \textbf{Text File Checkerboard:}

        Using the \texttt{checkerboard.txt} file (download at \texttt{https://tinyurl.com/ay7busxh}), draw a colored checkerboard that takes up the entire screen with the color of each square given in the file.

        Sample \texttt{checkerboard.txt}:\\
        \texttt{4\\255,255,255\\250,120,360\\80,80,80\\1,2,3}

        The first line of \texttt{checkerboard.txt} is the total number of boxes to draw.
        It is guaranteed that this number will be a perfect square.
        The following lines consist of the colors of all the boxes in the checkerboard.
        For a \texttt{checkerboard.txt} with first line $n$, create a $\sqrt{n} \times \sqrt{n}$ board using the given colors.

        For the sample given above, a correct solution to this problem would produce a $2 \times 2$ board with these color values:\\
        \texttt{color(255, 255, 255) | color(250, 120, 360)\\color(80,  80,  80) | color(1, 2, 3)}

        \textbf{Note:} The Processing function \texttt{sqrt()} may be useful.
    \end{numedquestion}
\end{document}